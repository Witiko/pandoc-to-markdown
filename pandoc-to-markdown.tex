% Pandoc to Markdown mapping
\def\pandocDocumentBegin{\markdownRendererDocumentBegin}%
\def\pandocDocumentEnd{\markdownRendererDocumentEnd}%
\def\pandocBlocksep{\markdownRendererInterblockSeparator}%

%% Block elements
\ExplSyntaxOn
\cs_new:Npn
  \pandocCodeBlock
  #1 #2
  {
    \tl_if_empty:nTF
      { #2 }
      { \markdownRendererInputVerbatim { #1 } }
      { \markdownRendererInputFencedCode { #1 } { #2 } }
  }
\cs_new:Npn
  \pandocRawBlock
  #1 #2
  {
    \tl_if_eq:nnT
      { #1 }
      { tex }
      { #2 }
  }
\ExplSyntaxOff
%%% Most block elements are broken into two "Begin" and "End" parts to avoid
%%% running out of memory during macro expansion, since their contents can be
%%% very large, especially in the case of Divs.
\def\pandocBlockQuoteBegin{\markdownRendererBlockQuoteBegin}%
\def\pandocBlockQuoteEnd{\markdownRendererBlockQuoteEnd}%
\def\pandocOrderedListBegin{\markdownRendererOlBegin}%
\def\pandocOrderedListItemBegin{\markdownRendererOlItem}%
\def\pandocOrderedListItemEnd{\markdownRendererOlItemEnd}%
\def\pandocOrderedListEnd{\markdownRendererOlEnd}%
\def\pandocBulletListBegin{\markdownRendererUlBegin}%
\def\pandocBulletListItemBegin{\markdownRendererUlItem}%
\def\pandocBulletListItemEnd{\markdownRendererUlItemEnd}%
\def\pandocBulletListEnd{\markdownRendererUlEnd}%
\def\pandocDefinitionListBegin{\markdownRendererDlBegin}%
\def\pandocDefinitionListItemBegin#1{\markdownRendererDlItem{#1}}%
\def\pandocDefinitionListDefinitionBegin{\markdownRendererDlDefinitionBegin}%
\def\pandocDefinitionListDefinitionEnd{\markdownRendererDlDefinitionEnd}%
\def\pandocDefinitionListItemEnd{\markdownRendererDlItemEnd}%
\def\pandocDefinitionListEnd{\markdownRendererDlEnd}%
\ExplSyntaxOn
\cs_new:Npn
  \pandocHeader
  #1 #2
  {
    \int_case:nn
      { #1 }
      {
        { 1 } { \markdownRendererHeadingOne { #2 } }
        { 2 } { \markdownRendererHeadingTwo { #2 } }
        { 3 } { \markdownRendererHeadingThree { #2 } }
        { 4 } { \markdownRendererHeadingFour { #2 } }
        { 5 } { \markdownRendererHeadingFive { #2 } }
        { 6 } { \markdownRendererHeadingSix { #2 } }
      }
  }
\cs_new:Npn
  \pandocHorizontalRule
  {
    \cs_if_exist_use:NF
      \markdownRendererThematicBreak
      \markdownRendererHorizontalRule
  }
\ExplSyntaxOff
\def\pandocDivBegin{}%
\def\pandocDivEnd{}%
\def\pandocDisplayMath#1{\[#1\]}%
\def\pandocTable{\markdownRendererTable}%

%% Inline elements
\def\pandocEmph{\markdownRendererEmphasis}%
\def\pandocUnderline#1{#1}%
\def\pandocStrong{\markdownRendererStrongEmphasis}%
\def\pandocStrikeout#1{#1}%
\def\pandocSubscript#1{#1}%
\def\pandocSuperscript#1{#1}%
\def\pandocSmallCaps#1{#1}%
\def\pandocSingleQuoted#1{`#1'}%
\def\pandocDoubleQuoted#1{``#1''}%
\def\pandocCite{\markdownRendererCite}%
\def\pandocCode{\markdownRendererCodeSpan}%
\def\pandocLineBreak{\markdownRendererLineBreak}%
\def\pandocInlineMath#1{$#1$}%
\ExplSyntaxOn
\cs_new:Npn
  \pandocRawInline
  #1 #2
  {
    \tl_if_eq:nnT
      { #1 }
      { tex }
      { #2 }
  }
\ExplSyntaxOff
\def\pandocLink#1#2#3#4{\markdownRendererLink{#1}{#2}{#3}{#4}}
\def\pandocImage#1#2#3#4{\markdownRendererImage{#1}{#2}{#3}{#4}}
\def\pandocCaptionedImage#1#2#3#4{\markdownRendererImage{#1}{#2}{#3}{#4}}
\def\pandocNote#1{#1}%
\def\pandocSpan#1{#1}%

%% Special characters
\def\pandocAmpersand{\markdownRendererAmpersand}%
\def\pandocBackslash{\markdownRendererBackslash}%
\def\pandocCircumflex{\markdownRendererCircumflex}%
\def\pandocDollarSign{\markdownRendererDollarSign}%
\def\pandocHash{\markdownRendererHash}%
\def\pandocLeftBrace{\markdownRendererLeftBrace}%
\def\pandocPercentSign{\markdownRendererPercentSign}%
\def\pandocPipe{\markdownRendererPipe}%
\def\pandocRightBrace{\markdownRendererRightBrace}%
\def\pandocTilde{\markdownRendererTilde}%
\def\pandocUnderscore{\markdownRendererUnderscore}%
